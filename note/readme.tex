\documentclass{article}
\usepackage{CJKutf8}
\usepackage{graphicx}
\usepackage{svg}
\raggedright
\begin{document}
\begin{CJK}{UTF8}{gkai}
\title{Active Worm}
\author{姜高晓}
\date{\today}
\maketitle
\begin{abstract}
这个程序包由两部分构成,模拟和分析,模拟部分是基于Java写的,分析部分是基于Python3写的。主要是研究里柔软的蠕虫,在形成簇的时候,簇自发旋转的一些性质.
\end{abstract}

\section{背景}
\subsection{活性物质}

	从Vicsek提出了Vicsek模型研究鸟群后,活性物质逐渐被生物物理学家关注,利用物理学方法分析细菌,生命系统的物理学性质,如活性物质中的有效温度,超扩散,粒子数巨涨落等。

	Vicsek 模型在速度上做了区域内平均,这一个效应可以认为是高等生命群体的群体智能。群体内部总是保持这一个运动方向,如大雁南飞。


	Active Brown Particles 模型给单个的球形粒子加了一个自驱动方向,这个模型往往去模拟细菌,但是自驱动的方向是一个角度扩散,这一点往往被人诟病
\section{程序介绍}
\subsection{模拟}
        这是内容i
\subsection{分析}
	这是模拟
\section{程序使用}
\subsection{Subsection title}
        这是总结




\begin{thebibliography}{12345}%引文
\bibitem{vicsek} Vicsek T
\bibitem{vicsek} Vicsek T
\bibitem{vicsek} Vicsek T


\end{thebibliography}
\end{CJK}
\end{document}




