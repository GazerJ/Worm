\documentclass{article}
%\usepackage{CJKutf8}
\usepackage{CJK}
\usepackage{indentfirst} 
\setlength{\parindent}{2em}
\begin{document}
%\begin{CJK}{UTF8}{gkai}
\begin{CJK}
\title{Active Worm}
\author{姜高晓}
\date{\today}
\maketitle
\begin{abstract}
这个程序包由两部分构成,模拟和分析,模拟部分是基于Java写的,分析部分是基于Python3写的。主要是研究里柔软的蠕虫,在形成簇的时候,簇自发旋转的一些性质.
\end{abstract}

\section{背景}
	从Vicsek提出了Vicsek模型研究鸟群后,活性物质逐渐被生物物理学家关注,利用物理学方法分析细菌,生
命系统的物理学性质,如活性物质中的有效温度,超扩散,粒子数巨涨落等。
\subsection{活性物质模型}
	Vicsek 模型在速度上做了区域内平均,这一个效应可以认为是高等生命群体的群体智能。群体内部总是保持这一个运动方向,如大雁南飞。

	Active Brown Particles 模型给单个的球形粒子加了一个自驱动方向,这个模型往往去模拟细菌,但是自驱动的方向是一个角度扩散,这一点往往被人诟病

	Active Worm 模型(本文),一条蠕虫身上由4个球形节点构成,由弹簧链接,每个球由LJ势能作为体积排斥是,也有弯曲能。头节点存在活性自驱动速度,方向是由尾巴指向头,大小跟周围密度成反比,这个模型可以很好的描述在集体运动的时候,个体之间活性运动方向的相互影响。
\section{程序介绍}
	程序由两部分构成,模拟部分和分析部分,下面分别讲解程序的组成和原理
\subsection{模拟}
        模拟程序主要源码由3个,置于/src/main/java/下,array.java 为控制台主程序,np.java为文件读写的程序,worm.java是核心的计算程序。
	
	array.java:每一个worm程序类指的是一次完整独立的实验,在研究问题时需要改变各种参数,进行多次实验,这里直接在array控制即可

	np.java:文件都读写。

	worm.java:核心类:worm,初始化参数为扩散系数,弯曲系数,活性自驱动速度。

\subsection{分析}
	这是模拟
\section{程序使用}
\subsection{Subsection title}
        这是总结




\begin{thebibliography}{12345}%引文
\bibitem{vicsek} Vicsek T
\bibitem{vicsek} Vicsek T
\bibitem{vicsek} Vicsek T


\end{thebibliography}
\end{CJK}
\end{document}




